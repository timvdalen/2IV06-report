%motivation of choices made

\subsection{Map file format}
When exporting our maps from the generator, all data remains intact. When importing the data structures that were present after generation are all restored. Alternatively we could have either put the data in a more commonly used format or we could have generated the geometry and merely exported that as an object file. The first alternative would in our case have unnecessarily complicated serialisation and deserialisation. The second alternative would have meant discarding much of the additional information concerning biomes and other intermediate data which could be interesting when the generated data is used in interactive applications.

\subsection{Terrain visualisation}
Within the limited time frame we chose to give rivers a geometric representation and focus on that instead of using texturing to indicate water. This posed interesting and tangible problems for us to solve while simultaniously turning out to be very challenging. Texturing the terrain appropriately posed other problems as the irregular polygon assignments made it very difficult to have the textures look seamless. Therefor the project does contain some code dealing with texturing but we decided to simply use biome appropriate coloring.

\subsection{Libraries}
We made both a map generation and viewing program and have used several libraries to provide the framework for our project.

\subsubsection{wxWidgets}
To provide the required functionality in the map generator specifically we decided to use the wxWidgets toolkit. This allows us to work towards a cross-platform solution and provides an easy way of implementing the necessary graphical user interface. To minimize the number of libraries are required to build both of the programs we intend to develop, we will also use wxWidgets to handle window setup for the map viewer. Even though it does not require most of the functionality provided by the library.

\subsubsection{OpenGL 4}
We wanted to make an effort to work with one of the most recent version of OpenGL. Partly because we were allready intending to write our own shaders to be able to have some more control over the eventual look of the generated landscape. The newer versions also provide some interesting types of shaders which may or may not be useful. Geometry and Tesselation shaders can actually alter the provided vertices which is interesting, but the downside is that we are trying to generate our maps before use instead of right before they are being shown to the user.

\subsubsection{Glew}
The wxGLCanvas provided by wxWidgets alone is not enough to be able to use our custom shaders. We will be using glew to take care of shader compilation and set up other advanced OpenGL features.